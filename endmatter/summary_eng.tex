%!TEX root = ../thesis.tex

\chapter*{Summary}
\markboth{Summary}{Summary}

% Add label in the list of contents
\addcontentsline{toc}{chapter}{Summary}

\begin{center}
	\textcolor{SchoolColor}{\Titulosize\bfseries Unraveling the elusive solvation structure of aqueous ions using advanced spectroscopic techniques} \normalsize \\
	\vspace*{9pt}
\end{center}


In this thesis, we investigate the solvation of ions in water using state-of-the-art spectroscopic techniques: GHz dielectric relaxation spectroscopy (DRS) and femtosecond time-resolved vibrational spectroscopy (TRVS), which are introduced in \textbf{Chapters 2} and \textbf{3}. In \textbf{Chapters 4}, \textbf{5}, and \textbf{6}, we demonstrate how DRS can be used to investigate solvation properties and the structure of water around ions. In \textbf{Chapters 7}, \textbf{8} and \textbf{9}, we apply TRVS to observe how the negatively charged oxygen atom of hydroxide and phenolate ions, which act as strong hydrogen-bond acceptors, affects the structural dynamics of the hydrogen-bond network of water.


In the experiments presented in \textbf{Chapter 4}, we study the extent to which the cooperative dynamics of water are affected by the presence of ions. Isotope-dependent measurements enable us to observe the dielectric response of water molecules surrounding ions. Our results indicate that these water molecules reorient in a much less cooperative manner than bulk water, as a consequence of the local disruption of the hydrogen-bond network by the ions. Our observations allow us to test for the first time and to empirically modify the Hubbard-Onsager model that quantifies the ion-induced dielectric deficiency of water. In contrast to the original model, the modified Hubbard-Onsager equation takes the disruption of the hydrogen-bond network by the ions into account, making it possible to determine hydration numbers in an unambiguous manner. These numbers are of paramount importance since they characterize the chemical and physical properties of aqueous solutions. 


In \textbf{Chapter 5}, the isotope effect is again used to determine hydration numbers of alkali-metal ions. Based on our observations, \ce{Na+} forms solvation shells of $\sim$7 water molecules, a hydration number slightly larger than the packing capacity of water in the first solvation shell. Upon increasing the size of the ion, the hydration number reduces to 4, 3 and 3 for \ce{K+}, \ce{Rb+} and \ce{Cs+}, respectively. In these three cases, the hydration number is smaller than the packing capacity, meaning that the first solvation shell is weakly hydrated. Our results are in line with the Hofmeister series which establishes the relative hydration ability of water depending on electrostatic ion--water interactions. 


Protons (\ce{H+}) and hydroxide ions (\ce{OH-}) are cases of special interest since they have the ability to form hydration complexes in which the excess charge is delocalized. In \textbf{Chapter~6}, we measure the isotope-dependent dielectric response to study the solvation of \ce{H+} and \ce{OH-} ions. From its hydration number, \ce{H+} is found to mostly exist in water as an Eigen H$_9$O$_4^+$ complex. From the study of the kinetic effects, we estimate the number of water molecules that are involved in the \ce{H+} and \ce{OH-} charge diffusion. This number is at least three times higher for \ce{H+} than for \ce{OH-}, meaning that the proton transfer mechanism is significantly different in acidic and alkaline solutions. 


In \textbf{Chapter 7}, we explore the extent to which \ce{OH-} ions affect the reorientation dynamics of water molecules. We examine whether \ce{OH-} can have an influence on the molecular dynamics of the solvent (water molecules beyond the first solvation shell). For solutions with \ce{OH-} concentration up to 4 molar, the reorientation dynamics of water are slightly slowed down. At higher \ce{OH-} concentrations, the remaining bulk-like solvent is cooperatively locked between neighboring \ce{OH-} ions due to a crowding effect. In this regime, the solution can be regarded as a semi-rigid hydrogen-bond network.


In \textbf{Chapter 8}, we continue our study of the structural properties of \ce{OH-} ions, but with a focus on the role that they could play in equilibrating vibrational excess energy of surrounding water molecules. We observe that hydroxide ions accelerate the relaxation of OD-stretch excitations of nearby HDO molecules. This effect is well-described through resonance energy transfer from HDO to \ce{OH-} with a F\"orster radius of 3~\AA, a value that matches the intermolecular distance in liquid water. Hence, \ce{OH-} ions act as a sink for excess vibrational energy of neighbouring excited water molecules. Our results suggest that \ce{OH-} ions may also participate in equilibrating excess energy in chemical reactions in alkaline environments.


Finally, in \textbf{Chapter 9}, we study how phenolate ions affect the reorientation dynamics of surrounding water molecules. Our results show that phenolate slows down the dynamics of the surrounding water molecules, an effect that extends even beyond the first solvation shell. This effect is due to the propensity of the negatively charged oxygen to form strong hydrogen bonds, which ``lock'' the hydrogen-bond structure.


The results presented in this thesis provide new information on the solvation properties of several commonly occupied ions. We hope that this thesis will inspire further experimental and theoretical studies, and may help to improve our understanding of the solvation structure and dynamics of ions in water.




