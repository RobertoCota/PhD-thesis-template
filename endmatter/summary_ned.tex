%!TEX root = ../thesis.tex

\chapter*{Samenvatting}
\markboth{Samenvatting}{Samenvatting}

% Add label in the list of contents
\addcontentsline{toc}{chapter}{Samenvatting}


\begin{center}
	\textcolor{SchoolColor}{\Titulosize\bfseries Het ontrafelen van de raadselachtige structuur van water rondom ionen met behulp van geavanceerde spectroscopische technieken} \normalsize \\
	\vspace*{9pt}
\end{center}


In dit proefschrift onderzoeken we hoe ionen oplossen in water, waarbij we gebruik maken van \textit{state-of-the-art} spectroscopische technieken: gigahertz di\"{e}lektrische relaxatiespectroscopie (DRS) en femtoseconde tijdsopgeloste vibrationele spectroscopie (TRVS). Deze technieken worden ge\"{i}ntroduceerd in \textbf{Hoofdstukken 2} en \textbf{3}. In \textbf {Hoofdstukken 4}, \textbf{5} en \textbf{6} laten we zien hoe DRS kan worden gebruikt om de eigenschappen van water rond ionen te onderzoeken. In \textbf{Hoofdstukken 7}, \textbf{8} en \textbf{9} passen we TRVS toe om de invloed van het negatief geladen zuurstofatoom in hydroxide- en fenolaat-ionen---beide sterke waterstofbrug-acceptoren---op de structuur en dynamica van het waterstofbrug-netwerk van water te bekijken.


In \textbf{Hoofdstuk 4} onderzoeken we de mate waarin de co\"{o}peratieve dynamica van water wordt be\"{i}nvloed door de aanwezigheid van ionen. Door isotoop-afhankelijke metingen uit te voeren kunnen we de di\"{e}lektrische respons van watermoleculen rondom ionen waarnemen. Uit onze resultaten blijkt dat deze watermoleculen op een veel minder co\"{o}peratieve manier reori\"{e}nteren dan bulkwater, als gevolg van de lokale verstoring van het waterstofbrug-netwerk door de ionen. Onze waarnemingen stellen ons in staat om het Hubbard-Onsager model voor het eerst experimenteel te testen. Dit model beschrijft de door ionen veroorzaakte vermindering van de di\"{e}lektrische respons van water. Uit de experimenten blijkt dat het Hubbard-Onsager model moet worden aangepast om een goede beschrijving te geven van water rondom ionen. In tegenstelling tot het oorspronkelijke model, houdt de aangepaste Hubbard-Onsager-vergelijking rekening met de verstoring van het waterstofbrug-netwerk door ionen. Hierdoor wordt het mogelijk om op een ondubbelzinnige manier het aantal hydraterende watermoleculen (kortweg, het ``hydratiegetal'') te bepalen. Dit aantal is van groot belang omdat het de chemische en fysische eigenschappen van waterige oplossingen karakteriseert.


In \textbf {Hoofdstuk 5} wordt het isotoop-effect opnieuw gebruikt om het hydratatiegetal van alkalimetaal-ionen te bepalen. Op basis van onze waarnemingen vormt \ce{Na+} solvatatieschillen van $\sim$7 watermoleculen, een aantal dat iets groter is dan de pakkingscapaciteit van water in de eerste solvatatieschil. Met toenemende ion-straal vermindert het aantal hydraterende watermoleculen tot respectievelijk 4, 3 en 3 voor \ce{K+}, \ce{Rb+} en \ce{Cs+}. In deze drie gevallen is het hydratatiegetal kleiner dan de pakkingscapaciteit, wat betekent dat de eerste solvatatieschil zwak gehydrateerd is. Onze resultaten komen overeen met de Hofmeister-reeks die de sterkte van de interactie tussen ionen en water voor verschillende ionen weergeeft.


Protonen (\ce{H+}) en hydroxide-ionen (\ce{OH-}) zijn bijzondere ionen, omdat ze het vermogen hebben om hydratatiecomplexen te vormen waarin de lading wordt gedelokaliseerd. In \textbf{Hoofdstuk~6} meten we de isotoopafhankelijke di\"{e}lektrische respons om de solvatatie van \ce{H+} en \ce{OH-} ionen te bestuderen. Uit het hydratatiegetal \mbox{blijkt} dat \ce{H+} meestal als een Eigen (H$_9$O$_4^+$) complex voorkomt in water. Door de di\"{e}lektrische respons in gewoon en zwaar water te vergelijken, kunnen we een schatting maken van het aantal watermoleculen dat betrokken is bij de ladingsdiffusie van \ce{H+} en \ce{OH-}. Dit aantal is minstens drie keer hoger voor \ce{H+} dan voor \ce{OH-}, wat betekent dat het transportmechanisme van protonen en hydroxide ionen aanzienlijk verschilt. Dit is verrassend omdat ook vaak is voorgesteld dat de transportmechanismes van deze twee ionen elkaars spiegelbeeld zouden zijn.


In \textbf{Hoofdstuk 7} onderzoeken we de mate waarin \ce{OH-} ionen de reori\"{e}ntatie van watermoleculen be\"{i}nvloeden. We onderzoeken of \ce{OH-} invloed heeft op de moleculaire dynamica van de watermoleculen buiten de eerste solvatatieschil. Voor oplossingen met een \ce{OH-} concentratie tot 4 molair wordt de reori\"{e}ntatiedynamica van water buiten de eerste solvatatieschil enigszins vertraagd, maar het effect is gering. Als de \ce{OH-} concentratie verder wordt verhoogd neemt dit effect aanzienlijk toe, het resterende water zit dan als het ware klem tussen de solvatatieschillen van naburige \ce{OH-}-ionen. In dit regime kan de oplossing worden beschouwd als een semi-rigide waterstofbrug-netwerk.


In \textbf{Hoofdstuk 8} bestuderen we de rol die \ce{OH-} ionen spelen bij het in evenwicht brengen van de overtollige vibrationele energie van omringende watermoleculen. We zien dat hydroxide-ionen de energierelaxatie van OD-strek excitaties van nabijgelegen HDO-moleculen versnellen. Dit effect wordt goed beschreven door resonante energieoverdracht van HDO naar de solvatatieschil van \ce{OH-}. Deze resonante energieoverdracht wordt gekarakteriseerd door een zogenaamde F\"{o}rsterradius van 3~\AA, een lengte die overeenkomt met de intermoleculaire afstand in vloeibaar water. Dit duidt er op dat de solvatatieschillen van \ce{OH-} ionen kunnen fungeren als een afvoer voor overtollige vibrationele energie van aangrenzende ge\"{e}xciteerde watermoleculen.


Ten slotte bestuderen we in \textbf{Hoofdstuk 9} hoe fenolaat-ionen de reori\"{e}ntatiedynamica van omringende watermoleculen be\"{i}nvloeden. Onze resultaten tonen aan dat fenolaat de dynamica van de omringende watermoleculen vertraagt, een effect dat zich zelfs tot voorbij de eerste solvatatieschil uitstrekt. Dit effect is een gevolg van de sterke waterstofbindingen die het negatief geladen zuurstof-atoom van fenolaat kan vormen, en waardoor de waterstofbrugstructuur als het ware ``vergrendeld'' wordt.


De resultaten in dit proefschrift geven nieuwe informatie over de solvatatie-eigenschappen van verschillende veel voorkomende ionen. We hopen dat dit proefschrift tot verdere experimentele en theoretische studies zal leiden, en ons begrip van de solvatatie-structuur en dynamica van ionen in water kan verbeteren.




