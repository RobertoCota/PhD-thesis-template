%!TEX root = ../thesis.tex

\chapter*{Resumen}
\markboth{Resumen}{Resumen}

% Add label in the list of contents
\addcontentsline{toc}{chapter}{Resumen}

\begin{center}
	\textcolor{SchoolColor}{\Titulosize\bfseries Entendiendo la misteriosa estructura molecular que rodea a los iones disueltos en agua mediante t\'ecnicas espectrosc\'opicas de vanguardia.} \normalsize \\
	\vspace*{9pt}
\end{center}


A lo largo de esta tesis se presenta un estudio sobre la solvataci\'on de iones con agua, usando para ello tecnolog\'ias espectrosc\'opicas de vanguardia como la espectroscop\'ia de relajaci\'on diel\'ectrica (ERD) con frecuencias en el orden de gigahercios (s\'uper altas frecuencias) y la espectroscop\'ia vibracional con una resoluci\'on temporal de femtosegundos (EVFS). Dichas t\'ecnicas son explicadas en los \textbf{Cap\'itulos 2} y \textbf{3}. Por otra parte, en los \textbf{Cap\'itulos 4}, \textbf{5} y \textbf{6} se muestra la forma de implementar la ERD para el estudio de las propiedades de solvataci\'on y el estudio de las estructuras moleculares del agua que rodea a los iones. Mientras que, en los \textbf{Cap\'itulos 7}, \textbf{8} y \textbf{9}, se usa EVFS para observar c\'omo el \'atomo de ox\'igeno que est\'a contenido en los iones de hidr\'oxido y fenolato, y que act\'ua como un aceptor de enlaces de hidr\'ogeno,  afecta las propiedades din\'amicas de la red de enlaces de hidr\'ogeno que forma el agua a nivel molecular. 


Con la serie de experimentos presentados en el \textbf{Cap\'itulo 4} se estudia la medida en que la din\'amica cooperativa de las mol\'eculas de agua cambia ante la presencia de iones. Con base a un an\'alisis de dos is\'otopos de agua, estudiamos de forma puramente experimental las propiedades diel\'ectricas de las mol\'eculas de agua en las proximidades de los iones. En \'esta regi\'on, las mol\'eculas de agua pierden su car\'acter cooperativo debido a la ruptura local de la red de enlaces de hidr\'ogeno. Adicionalmente, nuestros experimentos permitieron probar por primera vez la validez del modelo de Hubbard y Onsager, que predice la medida en que los iones, debido a su campo el\'ectrico, obstaculizan la movilidad de las mol\'eculas de agua. Nuestros experimentos dan lugar a modificaciones del modelo de Hubbard y Onsager, de tal forma que el efecto de la ruptura de la red de enlaces de hidr\'ogeno se puede tomar en cuenta, un efecto que es despreciado en el modelo original. Nuestro nuevo modelo permite determinar el n\'umero de hidrataci\'on de iones en agua de una forma f\'isicamente m\'as completa. Estos n\'umeros son de primordial importancia porque describen las propiedades qu\'imicas y f\'isicas de soluciones acuosas.



Usando el mismo an\'alisis del cap\'itulo anterior, en el \textbf{Cap\'itulo 5} se presenta un estudio para determinar los n\'umeros de hidrataci\'on de la familia de iones alcalinos. En este an\'alisis se observa que aproximadamente 7 mol\'eculas de agua forman un caparaz\'on o estructura de solvataci\'on alrededor de los iones de \ce{Na+}: este n\'umero est\'a por encima de la capacidad espacial de la primera capa de solvataci\'on. Sin embargo, al incrementar el tamaño del los iones, los n\'umeros de hidrataci\'on disminuyen a 4, 3 y 3 para \ce{K+}, \ce{Rb+} y \ce{Cs+}. En estos \'ultimos casos, los n\'umeros de hidrataci\'on son menores que la capacidad de la primera capa de solvataci\'on; lo cual indica que el efecto de hidrataci\'on es d\'ebil incluso en la vecindad inmediata de los iones. Estos resultados son coherentes con la serie de Hofmeister, donde se entiende el efecto de hidrataci\'on como el resultado de las interacciones electroest\'aticas entre los iones y las mol\'eculas de agua.



Los protones (\ce{H+}) y los iones de hidr\'oxido (\ce{OH-}) son casos de particular inter\'es por su habilidad de formar estructuras de hidrataci\'on donde la carga i\'onica est\'a delocalizada. En el \textbf{Cap\'itulo 6} se presenta un estudio de las propiedades de solvataci\'on de estos iones con la misma t\'ecnica de los dos cap\'itulos anteriores. El n\'umero de hidrataci\'on de los protones (\ce{H+}) revela que \'estos existen primordialmente en agua como estructuras ``Eigen’’, esto es H$_9$O$_4^+$. Por otra parte, a partir del estudio de los efectos cin\'eticos se puede estimar el n\'umero de mol\'eculas de agua que participan en el proceso de difusi\'on de la carga i\'onica. Este \'ultimo n\'umero es al menos tres veces mayor para protones que para iones de hidr\'oxido, lo que significa que el proceso de transferencia de protones es significativamente diferente en soluciones \'acidas que en soluciones alcalinas.



En el \textbf{Cap\'itulo 7} se presenta un estudio sobre la medida en que los iones de hidr\'oxido afectan la din\'amica de reorientaci\'on de las mol\'eculas de agua. En particular, examinamos si los iones de hidr\'oxido afectan la din\'amica del solvente (mol\'eculas m\'as all\'a de su primera capa de solvataci\'on). En concentraciones de hasta 4 molares, la din\'amica de reorientaci\'on molecular del agua es ligeramente desacelerada. Sin embargo, a concentraciones m\'as altas, la din\'amica de las mol\'eculas restantes fuera de cualquier caparaz\'on de hidrataci\'on se frena de manera cooperativa, atrapadas entre el efecto de m\'ultiples iones de hidr\'oxido. Este fen\'omeno es debido al efecto de la acumulaci\'on de iones. En este \'ultimo r\'egimen, se puede entender al sistema como una red semirr\'igida de enlaces de hidr\'ogeno. 



En el \textbf{Cap\'itulo 8} continuamos con el estudio de las propiedades estructurales de los iones de hidr\'oxido, pero ahora prestando particular atenci\'on al rol que podr\'ian tener en el equilibrio del exceso de energ\'ia vibracional existente en mol\'eculas circundantes. Nuestros resultados indican que los iones de hidr\'oxido aceleran el proceso de relajaci\'on vibracional de los enlaces de OD de las mol\'eculas de HDO cercanas. Adem\'as, nuestras observaciones son descritas mediante un proceso de transferencia de energ\'ia de resonancia desde las mol\'eculas de HDO hacia los iones de hidr\'oxido con un radio de acci\'on (radio de Förster) de 3~\AA, una distancia que coincide con el espaciamiento intermolecular en agua. As\'i pues, los iones de hidr\'oxido act\'uan como un pozo para el exceso de energ\'ia vibracional contenida en mol\'eculas excitadas circundantes. Estos resultados sugieren que los iones de hidr\'oxido pueden tambi\'en participar en el equilibrio de la energ\'ia excedente que pueda surgir de reacciones qu\'imicas en ambientes alcalinos.


Por \'ultimo, los experimentos presentados en el \textbf{Cap\'itulo 9} muestran c\'omo los iones de fenolato afectan la din\'amica de reorientaci\'on de las mol\'eculas de agua en su vecindad. Nuestros resultados tambi\'en demuestran que la din\'amica de las mol\'eculas circundantes es significativamente desacelerada, y que este efecto se extiende m\'as all\'a de su primera capa de solvataci\'on. Este fen\'omeno se explica debido a que el \'atomo de ox\'igeno del ion de fenolato, negativamente cargado, es susceptible a formar fuertes enlaces de hidr\'ogeno, que a su vez ``inmovilizan'' la estructura de enlaces de hidr\'ogeno del agua.


Los resultados presentados en esta tesis proveen nueva informaci\'on sobre las propiedades de solvataci\'on de m\'ultiples iones que son com\'unmente usados. Esperamos que el trabajo presentado en esta tesis sea capaz de inspirar futuros estudios, tanto experimentales como te\'oricos, y que logren mejorar la forma en que entendemos las estructuras de solvataci\'on y las propiedades de los iones disueltos en agua.





