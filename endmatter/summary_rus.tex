%!TEX root = ../thesis.tex

\chapter*{Заключение}
\markboth{Заключение}{Заключение}

% Add label in the list of contents
\addcontentsline{toc}{chapter}{Заключение}

\begin{center}
	\textcolor{SchoolColor}{\Titulosize\bfseries Изучение неуловимых структур гидратированных ионов с использованием современных спектроскопических методов} \normalsize \\
	\vspace*{9pt}
\end{center}


\hyphenation{мате-мати-ка восста-навливать вод-ных фем-то-се-кунд-ной яв-ляю-щи-е-ся ди-элек-три-че-ский по-лу-чен-ный экс-пе-ри-мен-тов де-фи-цит-ность коор-ди-на-ци-он-ные ослаб-ле-ние коор-ди-на-цин-ных огром-ное не-сколь-ко ко-ли-че-ство вра-ща-тель-ную со-дер-жа-щих воз-бу-жден-ных пере-рас-пре-де-ле-нии во-до-род-ных ин-фор-ма-цию меж-мо-ле-ку-ляр-ным хи-ми-че-ских nabij-gelegen mo-de-lo coope-ra-ti-va es-pe-ra-mos соль-ват-ную умень-ша-ет-ся обо-лоч-ки дис-сер-та-ци-ей}


В представленной работе проведено исследование сольватaции ионов в водных растворах с использованием современных спектроскопических методов: спектроскопии диэлектрической релаксации (СДР) в ГГц диапазоне и фемтосекундной времяразрешённой колебательной спектроскопии (ВРКС), которые описаны в \textbf{Главах 2} и \textbf{3}. В \textbf{Главах 4}, \textbf{5} и \textbf{6} продемонстрировано, как сольватационные свойства и структура воды вблизи ионов могут быть изучены при помощи СДР. В \textbf{Главах 7}, \textbf{8} и \textbf{9} при помощи ВРКС показано, как отрицательно заряженные атомы кислорода гидроксид и фенолят ионов, являющиеся сильными акцепторами водородной связи, влияют на структурную динамику системы водородных связей в воде. 

В экспериментах, представленных в \textbf{Главе 4} изучено, до какой степени коллективная динамика молекул воды подвержена влиянию ионов. Измерения с использованием различных изотопов позволили пронаблюдать диэлектрический отклик молекул воды, входящих в сольватную оболочку ионов. Полученный результат свидетельствует о том, что реориентация молекул воды, входящих в сольватную оболочку ионов, происходит под значительно меньшим влиянием кооперативных эффектов, чем в объеме воды, вследствие локальных разрывов в системе водородных связей под влиянием ионов. Наши наблюдения позволили впервые протестировать и модифицировать на основании экспериментов модель Хаббарда-Онзагера, описывающую диэлектрическую дефицитность воды, индуцированную ионами. В противоположность оригинальной модели, модифицированное уравнение Хаббарда-Онзагера учитывает ослабление водородных связей и позволяет таким образом однозначно определить координационные числа ионов. Данные координационные числа имеют огромное значение, так как они определяют химические и физические свойства водных растворов.

В \textbf{Главе 5} изотопный эффект снова использовался для определения координацинных чисел щелочных металлов. Исходя из наших наблюдений, катион Na$^+$ имеет сольватную оболочку, состоящую из  $\sim$7 молекул воды, что несколько больше, чем количество воды, способное уместиться в первую сольватную оболочку. При увеличении размера иона координационное число уменьшается до 4, 3 и 3 для K$^+$, Rb$^+$ и Cs$^+$ соответственно. В трёх последних случаях координационное число меньше, чем емкость первой сольватной оболочки, это значит, что ион слабо гидратирован. Наши результаты согласуются с лиотропным рядом Хоффмайстера, который описывает способность молекул воды гидратировать ионы, в зависимости от силы взаимодействий вода-ион.

Особый интерес представляют протоны (H$^+$) и гидроксид-ионы (OH$^-$) в силу того, что они могут образовывать гидраты, в которых избыточный заряд делокализован. В \textbf{Главе 6} для изучения сольватационных свойств ионов H$^+$ и OH$^-$ был измерен диэлектрический отклик в зависимости от изотопного состава растворов. Исходя из полученного значения координационного числа, гидратированный протон представляет из себя структуру типа катиона Айгена \ce{H9O4+}. По результатам изучения кинетических эффектов было оценено количество молекул воды, вовлеченных в диффузию зарядов H$^+$ и OH$^-$. Это количество как минимум в три раза выше для H$^+$, чем для OH$^-$, что указывает на существенное различие в механизме переноса протона в растворах кислот и щелочей. 

В \textbf{Главе 7} исследовано, насколько сильно OH$^-$ анионы влияют на вращательную динамику молекул воды. Мы изучили, может ли OH$^-$ влиять на молекулярную динамику растворителя (молекул, не входящих в первую сольватную оболочку). В растворах, с концентрацией OH$^-$ вплоть до 4~М вращательная динамика молекул воды незначительно замедлена. При больших концентрациях оставшиеся ``объёмные'' молекулы воды заблокированы между соседними OH$^-$ ионами вследствие стеснения ионов. В этом режиме раствор может считаться полужесткой системой водородных связей. 

В \textbf{Главе 8} продолжено исследование структурных свойств в системах, содержащих OH$^-$ ионы, однако акцент смещен в сторону изучения их роли в процессе достижения равновесия при колебательной релаксации соседних молекул воды. Показано, что гидроксид-ионы ускоряют релаксацию возбужденных OD-стретч колебаний близлежащих молекул \ce{HDO}. Этот эффект связан с Фёрстеровским резонантным переносом энергии от \ce{HDO} к OH$^-$  ионам с эффективным радиусом 3~\AA, это значение совпадает с межмолекулярным расстоянием в жидкой воде. Следовательно, OH$^-$ ионы выступают в качестве резервуара для избыточной колебательной энергии от соседних молекул. Эти результаты указывают на то, что OH$^-$ ионы могут участвовать в перераспределении энергии при достижении равновесия в процессе химических реакций в щелочной среде. 

Наконец, в \textbf{Главе 9} изучено влияние фенолят ионов на вращательную динамику молекул воды. Наши результаты демонстрируют, что присутствие фенолят-анионов приводит к замедлению вращательной динамики молекул воды, этот эффект распространяется дальше первой сольватной оболочки. Эффект связан со способностью атома кислорода фенолят-иона образовывать сильные водородные связи с молекулами воды, что “блокирует” систему водородных связей.

Результаты, представленные в данной диссертации содержат новую информацию о сольватации часто встречающихся ионов. Мы надеемся, что эта работа стимулирует новые экспериментальные и теоретические исследования и поспособствует лучшему пониманию сольватационных структур и динамики ионов в водных растворах. 






